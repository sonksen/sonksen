\documentclass[11pt,a4paper,notitlepage]{article}
\usepackage{graphicx,amsfonts, verbatim,amssymb,amsthm}
\usepackage{amsbsy,amsmath,setspace,natbib,psfrag}
\usepackage[letterpaper, left=1in, top=1in, right=1in,
bottom=1in,nohead, verbose, ignoremp]{geometry} 
%\usepackage[retainorgcmds]{IEEEtrantools}
\doublespace

\begin{document}
\title{Reference Posteriors From a Mixture of Reference Priors} 
\author{
    Michael D. Sonksen \\
    \small sonksen@stat.unm.edu \\
    \small Department of Mathematics and Statistics\\
    \small The University of New Mexico\\
    \small Albuquerque, NM 87131, USA
}
\maketitle

\begin{abstract}
In multi-parameter settings, the reference prior for a given
likelihood is not uniquely defined.  Instead, the reference prior
algorithm of Berger and Bernardo (1992) produces different reference
priors depending on a ordering and grouping of the model parameters.
In some instances an explicit formula for all possible reference prior
can be found.  We define a reference posterior using a mixture of all
possible reference priors.  We utilize the Dirichlet process to define
the prior distribution for the grouping and ordering of the
parameters.  The discrete nature of Dirichlet process priors makes it
an ideal mixing distribution.  Examination of the posterior
probabilities for the various models provides insight into the
viability of specific orderings and groupings.  We illustrate this
methodology and consider the associated computational issues with
a multinomial model and a constrained Poisson rate model.
\end{abstract}

%\begin{keyword}
%reference priors \sep Nonparametric Bayes \sep  Model Averaging \sep MCMC 
%\end{keyword}

\section{Introduction}

Reference priors, first conceived by \cite{Bernardo1979}, are among
the most popular non-informative priors used in practice.  In
multi-parameter problems, there is not one unique reference prior,
instead a reference prior is defined as function a specific group and
ordering of the parameters.  Because of the large number of possible
reference priors we consider an approach to define the reference
posterior as a mixture of all reference posteriors obtained from a
reference prior.  This is implemented by modeling the
grouping/ordering as a realization from a distribution following a
Dirichlet Process.

The rest of the paper is organized as follows. In Section 2, we review
the theory of reference priors and the dirichlet process.  In Section
3, we describe our proposed reference posterior as a mixture of
reference priors.  In Section 4, we examine theoretical properties of
this reference posterior.  In Section 5, we consider two examples. A
discussion of the proposed model and

\subsection{Reference Priors}
Reference priors are a class of non-informative priors which are an
extension of Jeffreys prior.  The theory and derivation of reference
priors was developed in a string of papers, namely \cite{Bernardo1979}
\cite{BergerBernardo1992b}, \cite{BergerBernardo1992}, and
\cite{BergerBernardoSun2009}.

\cite{Bernardo1979} defined the reference prior as a minimizer of
expected distance for one parameter or groups of parameters of
interest and nuisence parameters.  Bernardo's idea, is that the prior
distribution should impact the posterior distribution as little as
possible.  One way to define this is to define the reference prior as
the distribution for which the Expected Kullbach-Liebler divergence
(Equation~\eqref{EKL}) between the posterior distribution ($f$) and
prior distribution ($g$) is as large as possible (where the
expectation is taken with respect to the marginal distribution of the
data).
\begin{equation}
KL(f,g) = {\rm E}_{f}\left[\frac{f(\theta|X)}{g(\theta)}\right] \label{EKL}
\end{equation}

Unfortunately, the prior which minimizes Equation~\eqref{EKL} is often
discrete even for continuous parameters.  \cite{BergerBernardo1992b}
modified the definition of the reference prior to be the minimizer of
the expected KL divergence where the expectation is taken with respect
to repeated experiments $Z_1, Z_2, \ldots, Z_s$.  The minimizer for a
fixed $s$ is found and then passed to the limit.  The justification
for this is that the reference prior should be useful in repeated
experiments.

When the support of $\Theta$ is not compact, the expected K-L divergence 
may be infinity.  To overcome this problem,  \cite{BergerBernardo1992} 
recommend considereing a compact parameter space $\Theta_l$ such that 
\[
\lim_{l \to \infty} \Theta_l = \Theta.
\]
The minimizing function is then found for an arbitrary $l$ and a the
reference prior prior is taken as the limit as $l \to \infty$.

\cite{BergerBernardoSun2009} showed that a reference prior may also be
represented as a the maximizer of missing information.  They provide
an explicit form of the reference prior when there is only one parameter. 
Unfortunately, this work has yet to be extended to multiple parameters.

When more than one parameter exists, $\boldsymbol{\theta} = (\theta_1,
\theta_2, \ldots, \theta_k)$, the reference prior is simply Jeffreys
prior.  However, it is well known that in multi-parameter likelihoods,
Jeffreys prior can lead to poor posterior distributions (strong
inconsistency, impropriety).  To alleviate this problem,
\cite{BergerBernardo1992} and \cite{BergerBernardo1992b} defined the
reference prior as conditional on some grouping and ordering of the
parameters.  This sequential reference prior is defined by first
grouping the parameters into $m$ groups ordered groups
$\boldsymbol{\theta}_{(1)}, \boldsymbol{\theta}_{(2)}, \ldots,
\boldsymbol{\theta}_{(m)}$. Where group $i$ has $n_i$ number of
parameters. Further define $N_j = \sum_{i=1}^jn_i$. Let $S$ bet the
inverse of the Fisher Information matrix (orderd by the groups) and
$S_j$ is the upper left $N_j\times N_j$ upper left corner of $S$. If
$H_j=S_j^{-1}$ and $h_j$= lower right $n_j\times n_j$ corner of $H_j$
then the reference prior for this ordering and grouping is
\[
\pi^l(\boldsymbol{\theta}) = \left(\prod_{i=1}^m \frac{|h_i(\boldsymbol{\theta})|^{1/2}}{\int_{\Theta^l(\boldsymbol{\theta}_{[i-1]})}|h_i(\boldsymbol{\theta})|^{1/2}d\boldsymbol{\theta}_{(i)}}\right) 1_{\Theta^l}(\boldsymbol{\theta})
\]

\[
\pi(\boldsymbol{\theta}) = \lim_{l \to \infty} \frac{\pi^l(\boldsymbol{\theta})}{\pi^l(\boldsymbol{\theta^*})}
\]

if $|h_j|$ only depends on parameters from $\theta_{(1)},\ldots, \theta_{(j)}$.
Where $\boldsymbol{\theta^*}$ is any fixed point in $\Theta$. 

Obviously, this sequential defintion of the reference priors means
that there can be many different reference priors to use in a problem.
Pratically, users often try several different reference priors based
on intuative grouping/orderings and exaimine either theoretical
properties (prior moments, posterior coverage probabilities, etc) or
criterior based on a given sample (DIC, p-value diagnostics, Bayes
factors) to select a prior distribution for use.

In this work, we recommend averaging over all possible reference
priors.  To average over all possible reference priors, we will utized
the Nonparametric Bayes framework.  Specifically, we will consider a
Dirichlet Process mixture to model the ordering/grouping of our
parameters.


\subsection{Dirichlet Processes}

The Dirichlet process \citep{Fergeson, otherguy} is a commonly used
tool in the non-parametric Bayesian literature.  \cite{someguy} described
the dirichlet process as a probability distribution with support on
discrete probability functions.  Common descriptions of the Diriclet
process include stick breaking processes and the chineses restaurant
process.  We will not describe the Dirichlet process, and it's
properties in detail here, but instead refer the reader to \cite{dp1},
\cite{dp2}, and the referenences therein.  

The DP is characterized by it's base measure $G_0$ and mass parameter
$\alpha$.  We will represent such a process by $DP(\alpha, G_0)$. The
base measure $G_0$ defines the possible values and shape of
realizations from the DP.  The mass parameter, $\alpha$, affects the
variance of $\pi$ in the stick breaking representation.  A realization
from a DP process, $G|\alpha, G_0\sim DP(\alpha,G_0)$, is a discrete
probability distribution.

The realizations from a Dirichlet process are rarely used to model the
data directly.  Instead, parameters are often assumed to follow
realizations from a dirichlet process.  A classic example is the use
of the dirichlet process in hierarchical models.  Parameters,
representing latent group means, can be assumed to follow a DP.

The estimation of summaries of the posterior distribution of models
with a DP is often done with Markov chain Monte Carlo (MCMC) methods.  In
particular, the stick-breaking representation can be used to define a
Metropolis-Hastings algorithm with stationary distibution
approximately that of the posterior distribution.  In this work, we
use Algorithm 7 of \cite{Neal2006} for all computation.  For other
algorithms, see \cite{?}, \cite{??}, \cite{???}.


\section{Description of Methods}

\subsection{Ordering and Grouping as Latent Parameters}
Assume that all reference priors are proper or lead to a proper
posterior distribution.  The ordering and grouping of $k$ parameters
can be represented through a latent parameter vector
$\boldsymbol{\phi} = (\phi_1, \phi_2, \ldots, \phi_k)$. Each element
of this vector is tied to one of the parameters.  A larger $\phi_i$
implies that the parameter is of more inferential importance and when
$\phi_i = \phi_j$ the two parameters are of the same importance.  For
example, $\boldsymbol{\phi} = (1,1,4,2,2)$ would imply the grouping /
ordering: $\boldsymbol{\theta}_{1} = \left\{\theta_3\right\}$,
$\boldsymbol{\theta}_{(2)} = \left\{\theta_4,\theta_5\right\}$ and
$\boldsymbol{\theta}_{(3)} = \left\{\theta_1,\theta_2\right\}$.  This
represenation allows us to model the grouping/ordering by modeling
$\boldsymbol{\phi}$.  The choice of having larger $\phi$ denote a more
important parameter is arbitrary, but works well in our examples.

\subsection{Application to MDP}
To model $\boldsymbol{\phi}$, we assume that each component
independently follows a realization from the Dirichlet process with
base measure $F_0$ and mass parameter $\alpha$.  Because we do not
care about the distribution of $\phi$, outside of the ordering of
realizations, the choice of a continuous $F_0$ is arbitrary and can be
chosen for computational convience (in this work we use the standard
normal).  The discrete nature of realizations from a Dirichlet process
naturally will allow the elements of $\boldsymbol{\phi}$ to be equal
with positive probability.  The mass parameter $\alpha$, which in
essence controls the number of groups on average, can be fixed or
assumed to follow a distribution.  We consider both of these cases in
our examples in Section~\ref{sec:examples}.


Defining the reference prior as a mixture overall possible reference
priors allows the observed data to select the prior distribution which
provides a better fit.  

\subsection{Improper Reference Priors}
In the examples and derivations we consider in this work all of the
grouping and orderings yield prior distributions which correspond the
proper posterior distributions.  

When the parameter space is not compact, the reference priors could be
improper and there is a danger that the resulting posterior
distribution will be improper.

In this case, we propose two options to avoid this disasterous
problem.  

If we can identify which grouping/orderings induce a reference prior
which yields an improper posterior, those grouping/orderings can be
assigned a prior probability of zero.  

The latter situation may not be feasible, especially if a large number
of grouping/orderings exist.  

Instead, one can utilize a common assumption of the reference prior
derivation for non-compact parameter spaces.  

In the reference prior formula, the reference prior,
$\pi_l(\boldsymbol{\theta})$ is derived for a compact subspace
$\Theta_l$ such that $\lim_{l \to \infty}\Theta_l = \Theta$.  As an
approximation, we can average over all reference priors for a given
$l$ and use the posterior probabilities of the ordering/groupings to
identify potential reference priors for use.  Of course one would have
to check if the recommended reference priors induce a proper posterior
distribution.


Another potential concern is that the reference priors may not be of
closed form or easily found.  In these cases the reference priors 
can be approximated using the algorithm in Section 3.2 of \cite{BergerBernardo1994}.


\subsection{Computation} \label{sec:comp}



The model may be fit using the algorithms of \cite{Neal2000}.

\section{Properties}

-Minimizing KL Divergence (long shot)\\

When considering non-parameteric mixtures of priors, the is often an
interest in showing that the resulting posterior distribution is consistent. \\

-Posterior Consistency (prove!!!!)\\



\section{Examples}\label{sec:examples}
We will consider three examples to illustrate the proposed methodology.
The first utilizes a simulations study around the multinomial
distribution to invistigate small sample properties of the proposed
models.  We also examine a constrained Poisson rate model which is
used to estimate mortality rates from a well known actuarial dataset. 

\subsection{Multinomial Model}
A likelihood for which the form of all reference priors is known is the
multinomial model \citep{BergerBernardo1992b}.  If we assume that there are
$k$ parameters, the likelihood for $\boldsymbol{Y} = (Y_1, Y_2,\ldots, Y_k)$ when $N\le\sum_{i=1}^kY_i$ is known
is
\[
f(\boldsymbol{Y}|\boldsymbol{\theta}) = {N \choose Y_1,Y_1,\ldots, Y_n}\prod_{i=1}^k\left(1-\sum_{i=1}^k\theta_i\right)^{N-\sum_{i=1}^kY_i}\prod_{i=1}^k\theta_i^{Y_i}.
\]
\cite{BergerBernardo1992b} showed that the ordered-group reference prior has the form: 
\[
\pi_{\tt multinom}(\boldsymbol{\theta}) \propto \prod_{i=1}^{k}
\]

We will average over all possible reference priors by allowing the
parameters ($\boldsymbol{\phi}$) indexing the ordered groups to follow
realizations of the Dirichlet Process with a standard normal base
measure.  For the mass parameter of the Dirichlet Process, we will
consider four different values (0.01,0.1,1.0,10.0).  This makes our
model for a given $\alpha$:

\begin{eqnarray*}
  F_0 &=& N(0,1),\\
  F &\sim& DP(F_0, \alpha),\\
  \phi_i|F &\stackrel{iid}{\sim}& F,\\
  \boldsymbol{\theta}|\boldsymbol{\phi} &\sim&  
  \pi_{\rm multinom}(\boldsymbol{\theta}|\boldsymbol{\phi}), \\
  Y_i | \theta_i, N_i &\stackrel{ind}{\sim}& Multinomial(N,\boldsymbol{\theta}).
\end{eqnarray*}

To investigate properties of this mixture of reference priors model,
we performed the following simulation study.  We examine different
values of $k$ and fixed $\boldsymbol{\theta}$.  To investigate the
small sample properties, we consider the frequentist coverage
probabilities of the resulting 95\% posterior intervals with 1000
intervals compared for each set of parameters.  The posterior
intervals were estimated from quantiles of posterior samples generated
the Metropolis-Hasting algorithm described in Section~\ref{sub:comp}.

The estimated frequestist coverage probabilities are displayed in 
Table~\ref{tab:freq}.

%\begin{table}{}
\begin{tabular}{c|cccccc}
\hline
N & $\theta_1$&$\theta_2$&$\theta_3$&$\theta_4$&$\theta_5$&$\theta_6$\\
\hline
\end{tabular}
%\end{table}

\subsection{Constrained Poisson Rate Model}
\cite{SonksenPeruggia2012} provides another model where the reference
prior for any grouping/ordering is known.  They model the mortality data 
of \cite{Broffitt1988} by assuming that the number of observed deaths 
($Y_i$) in $N_i$ total people of age $i$ follows a Poisson distribution. 
\begin{eqnarray}
Y_i | \psi_i, N_i \stackrel{ind}{\sim} Poisson(N_i\psi_i)\label{model:poi}
\end{eqnarray}

Where $\psi_i$ is the mortality rate of age group $i$ for $i = 35,
36, \ldots, 64$.  The observed mortality rates ($Y_i/N_i$) are
displayed in Table~\ref{tab:poi_data}.  

\begin{table}[!tb] 
  \caption{Insurance records on mortality data for male individuals in 
    30 age groups \citep{Broffitt1988}. For each age group, the table 
    reports the total number of individuals and the observed number of 
    deaths. If an individual joined or left the insurance policy during 
    a given year, he is counted as 0.5 of a person.} \label{tab:poi_data}
\begin{center}
\begin{tabular}[c]{|rrr|rrr|rrr|}
\hline
age & size & deaths & age & size & deaths  & age & size & deaths  \\
\hline
35 & 1771.5 & 3 & 45 & 1931.0 & 8 & 55 & 1204.5 & 11     \\
36 & 2126.5 & 1 & 46 & 1746.5 & 13 & 56 & 1113.5 & 13          \\
37 & 2743.5 & 3 & 47 & 1580.0 & 8 & 57 & 1048.0 & 12            \\
38 & 2766.0 & 2 & 48 & 1580.0 & 2  & 58 & 1155.0 & 12         \\
39 & 2426.0 & 2 & 49 & 1467.5 & 7  & 59 & 1018.5 & 19        \\
40 & 2368.0 & 4 & 50 & 1516.0 & 4  & 60 & 945 & 12         \\
41 & 2310.0 & 4 & 51 & 1371.5 & 7 & 61 & 853 & 16    \\
42 & 2306.5 & 7 & 52 & 1343.0 & 4 & 62 & 750 & 12      \\
43 & 2059.5 & 5 & 53 & 1304.0 & 4  & 63 & 693 & 6   \\
44 & 1917.0 & 2 & 54 & 1232.5 & 1 & 64 & 594 & 10     \\
\hline 
\end{tabular}
\end{center}
\end{table}

\cite{SonksenPeruggia2012} showed that the reference prior, for a
given ordered group $\boldsymbol{\theta}_{(1)}, \ldots,
\boldsymbol{\theta}_{(m)}$, of the model in
Expression~\eqref{model:poi} satisfies.

\begin{equation}
\pi_{\rm poi\_ref}(\boldsymbol{\theta}|\boldsymbol{\phi})  
%\propto  
%\frac{\prod_{i=1}^k |h_i(\theta)|^{1/2}}{\prod \int_{
%\Theta_{(i)} | \Theta_{(1:(i-1))}} d\theta_{(i)} |h_i(\theta)|^{1/2}} 
 \propto  \frac{1}{\prod_{i=1}^{k}\sqrt{\theta_i}} \times 
\frac{1}{\prod_{j=2}^{m}(\sqrt{\gamma_j} - 
\sqrt{\eta_{j}})^{n_j}}  \times I_{\Theta_{\rm
  Incr}}(\boldsymbol{\theta}),
\label{eq:ref_incr}
\end{equation}
where,
\[
\gamma_{j+1} = \left\{\begin{array}{ll}
\min[\boldsymbol{\theta}_{(1:j)} : \boldsymbol{\theta}_{(1:j)}>\max(\boldsymbol{\theta}_{(j+1)})], & \rm{if} \,\,\, 
\max[\boldsymbol{\theta}_{(1:j)}] > 
%\theta_{(j+1)_{m_{(j+1)}}} =
\max[\boldsymbol{\theta}_{(j+1)}], \\
u, &  \rm{if} \,\,\, 
\max[\boldsymbol{\theta}_{(1:j)}] < 
%\theta_{(j+1)_{m_{(j+1)}}},\\
\max[\boldsymbol{\theta}_{(j+1)}], \\
\end{array}
\right.
\] 
and 
\[
\eta_{j+1} = \left\{\begin{array}{ll}
\max[\boldsymbol{\theta}_{(1:j)}:\boldsymbol{\theta}_{(1:j)}<\min(\boldsymbol{\theta}_{(j+1)}) ], & \rm{if} \,\,\, 
\min[\boldsymbol{\theta}_{(1:j)}] < 
%\theta_{(j+1)_1}, \\
\min[\boldsymbol{\theta}_{(j+1)}], \\
0, &  \rm{if} \,\,\, 
\min[\boldsymbol{\theta}_{(1:j)}] > 
%\theta_{(j+1)_1}.\\
\min[\boldsymbol{\theta}_{(j+1)}]. \\  %fix
\end{array}
\right.
\] 

To complete the model specification, we assume that the measure $F_0$ is
a standard normal and that the mass parameter $\alpha$ follows an
exponential distribution with mean of 1.  Making our final model:
\begin{eqnarray*}
\alpha &\sim& Exp(1),\\
F_0 &=& N(0,1),\\
F &\sim& DP(F_0, \alpha),\\
\phi_i|F &\stackrel{iid}{\sim}& F,\\
\boldsymbol{\theta}|\boldsymbol{\phi} &\sim&  \pi_{\rm poi\_ref}(\boldsymbol{\theta}|\boldsymbol{\phi}), \\
Y_i | \theta_i, N_i &\stackrel{ind}{\sim}& Poisson(N_i\theta_i).
\end{eqnarray*}

We utilized the algorithm described in Section~\ref{sec:comp} to
obtain samples from the posterior distribution of all parameters.
Figure~\ref{fig:poi} displays in red the estimated posterior means (solid
lines), with 95\% posterior intervals (dashed line), for all of the
mortality rates.  The posterior means and 95\% intervals for the
recommended reference prior in \cite{SonksenPeruggia2012} are
displayed in green.  We see that, at least visually, the DP mixture of
reference priors provides a better fit.  In terms of DIC, the DP
mixture of reference priors also outperforms the recommended model.

An interesting exercise is to examine which grouping and orderings have
the highest posterior probability.  This is allows us to see which
reference priors the data prefers and can give us some insight into
what the grouping and ordering mean in the reference prior algorithm.
Table~\ref{tab:poi_rank} displays the ten groupings and orderings which have
the highest posterior probability.  It is of note that the the third
most probable model is the one Sonksen and Peruggia (2012) selected
through intuition and DIC.

\begin{table}
\begin{center}
\begin{tabular}{c|c|c|c|c}
%&Ordering/Grouping & &&Rank\\
$\boldsymbol{\theta}_{(1)}$&$\boldsymbol{\theta}_{(2)}$&$\boldsymbol{\theta}_{(3)}$&$\boldsymbol{\theta}_{(4)}$ & Rank \\
\hline
 $\{\theta_{1}\}$& $\{\theta_{2},\ldots,\theta_{29}\}$&$\{\theta_{30}\}$  & &1\\
$\{\theta_{1}\}$& $\{\theta_{30}\}$& $\{\theta_{2},\ldots,\theta_{29}\}$ & &2\\
$\{\theta_{30}\}$& $\{\theta_{1}\}$& $\{\theta_{2},\ldots,\theta_{29}\}$ & &3\\
$\{\theta_{1}, \theta_{30}\}$&$\{\theta_{2},\ldots,\theta_{29}\}$& & &4\\
 $\{\theta_{30}\}$& $\{\theta_{2},\ldots,\theta_{29}\}$&$\{\theta_{1}\}$  & &5\\
  $\{\theta_{2},\ldots,\theta_{29}\}$&$\{\theta_{1}\}$&$\{\theta_{30}\}$  & &6\\
$\{\theta_{30}\}$&  $\{\theta_{1}, \ldots, \theta_{29}\}$ & & &7\\
$\{\theta_{2}\}$&$\{\theta_{1}\}$&  $\{\theta_{3}, \ldots, \theta_{29}\}$&$\{\theta_{30}\}$&8\\
$\{\theta_{1}, \ldots, \theta_{29}\}$&$\{\theta_{30}\}$& &&9\\
$\{\theta_{9}\}$&$\{\theta_{1}\}$&$\{\theta_{2}, \ldots, \theta_{8}, \theta_{10}, \ldots, \theta_{29}\}$&$\{\theta_{30}\}$& 10\\
\end{tabular}
\caption{Posterior Rankings}\label{tab:poi_rank}
\end{center}
\end{table}



\subsection{One-Way Balanced Random Effects Model}

As an example, consider a standard One-Way, balanced, random effects model:
\begin{eqnarray*}
y_{ij} &=& \mu + \alpha_{i} + \epsilon_{ij}\label{norm}\\
\alpha_{i}|\tau^2&\stackrel{iid}{\sim}& N(0,\tau^2)\\
\epsilon_{ij}|\sigma^2&\stackrel{iid}{\sim}& N(0,\sigma^2)
\end{eqnarray*}

for $i = 1, 2, \ldots n$, $j=1, 2, \ldots, k$. Since the random effects
($\boldsymbol{\alpha}=(\alpha_1, \ldots, \alpha_k$) are by design
normally distributed, there are only 3 parameters in need of a prior.
The group mean $\mu$, the error variance $\sigma^2$, and the random
effect variance $\tau^2$.

In this model, there is only 13 possible groupings and orderings. All
of which yield a proper posterior
distribution. \cite{BergerBernardo1992b} gives the form of the
reference priors for each possible ordering and grouping.

\[
\pi(\mu,\sigma^2,\tau^2|\phi) \propto \begin{cases}
  \sigma^{-2}\left(n\tau^2+\sigma^2\right)^{-3/2} & \text{if }\phi=(1,1,1)\\
  \sigma^{-5/2}\left(n\tau^2+\sigma^2\right)^{-1} & \text{if }\phi=(1,1,2)\\
  \tau^{-3C_n/2}\sigma^{-2}\psi(\tau^2/\sigma^2)& \text{if }\phi=(1,2,1)\\
\sigma^{-1}\left(n\tau^2+\sigma^2\right)^{-3/2} & \text{if }\phi=(2,1,2)\\
  \tau^{-1}\sigma^{-2}\left(n\tau^2+\sigma^2\right)^{-1/2}\psi(\tau^2/\sigma^2) & \text{if }\phi=(2,2,1)\\
  \sigma^{-2}\left(n\tau^2+\sigma^2\right)^{-1} & \text{if }\phi=(1,2,2), (2,1,1), (1,2,3), (2,1,3), (3,1,2)\\
  \tau^{-C_n}\sigma^{-2}\phi(\tau^2/\sigma^2) & \text{if }\phi=(1,3,2),(2,3,1),(3,2,1)
\end{cases}
\]
With $C_n = 1-\sqrt{n-1}\left(\sqrt{n}+\sqrt{n-1}\right)^{-3}$,
$\psi(\tau^2/\sigma^2)=\left((n+1)+\left(1+
    n\tau^2/\sigma^2\right)^{-2}\right)^{1/2}$ for $i=1,2,\ldots, p$
and $j=1,2,\ldots, n$. See \cite{Ye1991} for an alternative reference
prior based on a reparameterization.

For data, we set $n=10$ and $m=3$ and generated $\alpha$ and $y$ from
the model in Equation~\eqref{norm} setting $\mu=2$, $\sigma^2=16$ and
$\tau^2=4$. 


\section{Discussion}


\pagebreak
\bibliographystyle{apalike}

\bibliography{bib}

\end{document}

